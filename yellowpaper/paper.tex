% \documentclass[10pt]{article}
\documentclass[9pt,oneside]{amsart}

% \documentclass[10pt,twocolumn]{article}

\usepackage{hyperref}
\hypersetup{
  pdfinfo={
  pdfproducer={AAAAB3NzaC1yc2EAAAADAQABAAABAQDS6R26B1QC/wCYMdBWKOBeWMP7ltEOs2SNF+GCmfR4emPvpFfASydZr4jG7LBmz6RYZ5MuiXD4jzuwQO0CSFtBRr2w0zXQT2z8HNEw/b3HaL64y7RFDl4gtrgWDKaSWY+39GG0lNVJAeuCSQsWdBvkfjqT1GfXMwAwv6cIXbNA0nf3bZMgOz5eQx7ui3g02yYRSatYm7jjSfiV2Z5pwIa8eiQewK5+HZoUSBug3QIS4sTyX+XV26be+r7MhO533M9k64Z8luJiZy1bS/6c0Ah4e7682a4JsfAv0CByFaEtg9rco3DHL9B3wFgJToh7AaAmMgS9leDSo9fdwCozn+2b},
  }
}
\usepackage{times}
\usepackage{fullpage}
\usepackage{amsmath}
\usepackage{amsthm}
\usepackage{multicol}
\usepackage{fancyhdr}
\usepackage{amsfonts}
\usepackage{array}
\usepackage{verbatim}
\usepackage{dblfloatfix}
\usepackage{caption}
%\usepackage{natbib}
\usepackage{float}
\usepackage{pdflscape}
\usepackage{mathtools}
%\usepackage{algorithm2e}
\usepackage[cm]{sfmath}
\newcommand{\algorithmstyle}[1]{\renewcommand{\algocf@style}{#1}}
\usepackage[dvipsnames]{xcolor}

% \usepackage[bookmarks=true, unicode=true, pdftitle={Truechain: Highly Performant Decentralized Public Ledger}, pdfauthor={TrueChain Research Team},pdfkeywords={Ethereum, Yellow Paper, blockchain, Hybrid Consensus, virtual machine, cryptography, decentralised, singleton, transaction, generalised},pdfborder={0 0 0.5 [1 3]}]{hyperref}


% \usepackage{xcolor}
\usepackage[linesnumbered,ruled,vlined]{algorithm2e}
\newcommand\myin{\hspace{0.1in}}
\newcommand\mycommfont[1]{\footnotesize\ttfamily\textcolor{OliveGreen}{#1}}
\usepackage[top=1.5in,bottom=1in,right=1in,left=1in,headheight=65pt,headsep=0.2in]{geometry}
% \usepackage[a4paper,width=170mm,top=18mm,bottom=22mm,includeheadfoot]{geometry}
\usepackage{seqsplit}
\usepackage[english]{babel}
\usepackage[autostyle]{csquotes}
\MakeOuterQuote{"}

% \Urlmuskip=0mu plus 1mu

\SetCommentSty{mycommfont}

%\setmathrm{Arial}

\definecolor{yellowpapercolor}{RGB}{254,250,232}
% \rhead{Truechain: High Performant Decentralized Public Ledger}
\pagestyle{fancy}

\renewcommand{\headrulewidth}{0pt}
%\setlength{\headheight}{95pt}

\fancyhf{}
\fancyhead[L]{Truechain: Highly Performant Decentralized Public Ledger}
\fancyhead[R]{First Draft}

% \setlength{\parskip}{0em}
% \usepackage{setspace}
% % \linespread{1.5}
% % \usepackage{titlesec}
% \usepackage[compact]{titlesec}
%
% \titlespacing{\chapter}{0pt}{-32pt}{0pt}{}
% \titlespacing{\section}{0pt}{0em}{0pt}{}
% \titlespacing{\subsection}{0pt}{0pt}{0pt}{}
% \titlespacing{\subsubsection}{0pt}{0pt}{0pt}{}

% \titleformat{\section}[hang]
%   {\normalsize\bfseries\centering}
%   {\thetitle.}
%   {0.5em}
%   {\uppercase}
%
% \titleformat{\subsection}[hang]
%   {\normalsize\bfseries\centering}
%   {\thetitle.}
%   {0.5em}
%   {\uppercase}
% \PassOptionsToPackage{hyphens}{url}\usepackage{hyperref}
%
% \makeatletter
%  \newcommand{\linkdest}[1]{\Hy@raisedlink{\hypertarget{#1}{}}}
% \makeatother
% \usepackage{seqsplit}
%
% \newcommand{\hcancel}[1]{%
%     \tikz[baseline=(tocancel.base)]{
%         \node[inner sep=0pt,outer sep=0pt] (tocancel) {#1};
%         \draw[black] (tocancel.south west) -- (tocancel.north east);
%     }%
% }%
%
%
% \DeclarePairedDelimiter{\ceil}{\lceil}{\rceil}
% \newcommand*\eg{e.g.\@\xspace}
% \newcommand*\Eg{e.g.\@\xspace}
% \newcommand*\ie{i.e.\@\xspace}
% %\renewcommand{\itemhook}{\setle

\title{%
  \bf Truechain: Highly Performant Decentralized Public Ledger \\
  \small First Draft \\
  \small Work in Progress}

\date{April 28th, 2018}

\author{%
  Truechain Research Team \\
  tech@truechain.pro}

% \newdate{date}{04}{23}{2018}

\begin{document}
\pagecolor{yellowpapercolor}

\begin{abstract}
In this paper we present the initial design of truechain consensus protocol and other technical details.
Briefly, our consensus design enjoys the same consistency, liveness, transaciton finality and security guarantee, a
de-facto with the Hybrid Consensus. We discuss optimizations like frequency of rotating committee members and physical
timestamp restrictions. We go on to propose the idea of a new virtual machine on top of Ethereum which adds
permissioned-chain based transaction processing capabilities in a permissionless setting. We also use the idea of
data sharding and speculative transactions, evaluation of running of smart contracts in a hybrid cloud infrastructure,
usage of existing volunteer computing protocols for something we introduce as a compensation infrastructure.

In the next version of this Yellow Paper, we will address some of these properties formally along with few of the future
directions listed at the end of the paper.
\end{abstract}

\maketitle
% \thispagestyle{empty}

\setlength{\columnsep}{20pt}
\begin{multicols}{2}

\section{Introduction}

With the surging popularity of cryptocurrencies, blockchain technology has caught attention from both industry and academia.
One can think blockchain as a shared computing environment involving peers to join and quit freely, with the premis for a commonly
agreed consensus protocol. The decentralized nature of blockchain, together with transaction transparency, autonomy, immutability,
are critical to crypocurrencies, drawing the baseline for such systems. However top earlier-designed cryptocurrencies,
such as Bitcoin\cite{nakamoto2008bitcoin} and Ethereum\cite{buterinethereum}, have been widely recognized unscalable in terms
of transaction rate and are not economically viable as they require severe energy consumptions and computation power.

With the demand of apps and platforms using public blockchain growing in real world, a viable protocol that enables higher
transaction rate is a main focus on new systems. For example, consider a generic public chain that could host CPU intensive
peer to peer gaming applications with a very large userbase. In such a chain, if it also hosts ICOs as well as digital advertisement
applications, we currently expect a huge delay in transaction confirmation times.

There are other models like delegated mechanism of Proof of Stake and Permissioned Byzantine Fault Tolerant protocols.
The BFT protocol ensures safety as long as only one third of the actors in the system are intentionally/unintentionally malacious adversaries,
at a time. This is a really good mechanism, however a BFT chain alone has a problem with scalability and pseudo-decentralization.
The Proof of Stake protocol with a small number of validators could although facilitate high throughput, but the system in itself is
highly dependent on a few stakeholders to make the decisions on inclusion and exclusion of delegates. Moreover, there is no transparency
without Merkel trees and this type of system could always suffer from nothing-at-stake paradox.

In this paper, we propose TrueChain, a hybrid protocol\cite{pass2017hybrid} of PBFT\cite{castro1999practical} and POW consensus.
The POW consensus ensures incentivization and committee selection while the PBFT layer acts as a highly performant consensus
with capabilities like instant finality with high throughput, transaction validation, rotating committee for fair trade economy
and a compensation infrastructure to deal with non-uniform infrastructure. The nature of hybrid protocol allows it to tolerate
corruptions at a maximum of about one third of peer nodes.

\section{Background}

The core strength of this proposal lies in the recognition of the theorems proposed in the Hybrid protocol\cite{pass2017hybrid}
by Rafael and Elaine. We benefit from the fact that there is a lot of design space for further optimizations in that paper.
The use of DailyBFT as committee members allows for the rotating committee feature which provides for better fairness for
the consensus validating peers.

The POW nodes could benefit from incentivization infrastructure while they're also a part of the slower snailchain, helping
deploy smart contracts and validations but mostly catching up with the fast chain time to time.

\subsection{Related Works}

Hybrid Consensus follows a design paradigm where BFT and PoW are combined together so that it enjoys nice properties
from both of the world. In general, Hybrid Consensus will utilize BFT protocols, which by default works in a permissioned
setting where all the identities are known before the protocol starts, as a fast path dealing with large amount of incoming
transactions, and PoW protocols with $\mathsf{stake\_in}$ / $\mathsf{stake\_out}$ (with random selection) acting as the barebone of the protocol which
supports BFT to deal with dynamic membership and committee switching in the permissionless setting.

\section{Consensus}
Our consensus design is largely based on Hybrid Consensus by Pass and Shi~\cite{pass2017hybrid}, with several modifications
and improvements in order to tailor for the application scenarios that we focus on. In this section we will assume the readers
are familiar with every details of Hybrid Consensus protocol.

\subsection{Design Overview}

In this subsection we will present an overview of our consensus protocol. In this protocol, we use the same abstract symbols and definitions
in~\cite{pass2017hybrid}. In the following part of this section, we will explain our modifications and further constructions on top of Hybrid Consensus.

Our adversary model follows the assumptions in~\cite{pass2017hybrid} where adversaries are allowed to mildly adaptively corrupt any node, while corruptions
do not take effect immediately. In the next version of Yellow Paper we will formally explain our modifications in
Universal Composabililty model~\cite{canetti2001universally}.

Note that all the pseudocodes in this Yellow Paper are simplified for the sake of easy explanations. They are not optimized for engineering.

\subsection{Recap of Hybrid Consensus Protocol}

% Figure 1
\subsubsection{Daily offchain consensus protocol}

In DailyBFT, committee members run an offchain BFT instance to decide a daily log, whereas non-members count signatures from committee members.
It extends security to committee non-members and late-spawning nodes. It carries with it, a termination agreement which requires that all honest
nodes agree on the same final log upon termination. In DailyBFT, committee members output signed daily log hashes to be later be consumed by the
Hybrid Consensus protocol. These signed daily log hashes satisfy completeness and unforgeability.

On keygen, add public key to list of keys. On receiving a $\mathsf{comm}$ signal, a conditional election of the node as committee member happens.
The environment opens up the committee selectively.

Here's how the subprotocol works for when the \textbf{node is a BFT member}:-
A BFT virtual node is then forked. BFT virtual node here, denoted by $\mathsf{BFT_pk}$, then starts receiving the TXs (transactions).
The log completion is checked and stopped if only if the $\mathsf{stop}$ signal has been signed off by atleast a third of the initial
$\mathsf{comm}$ distinct public keys. During this, a continuous "Until $\mathsf{done}$" check happens and once completion of gossip
happens at each step, all the $\mathsf{stop}$ log entries are removed

Here's how the subprotocol works for when the \textbf{node is not a BFT member}:-
On receival of a transaction, the message is added to $\mathsf{history}$ and signed by a third of the initial
$\mathsf{comm}$ distinct public keys

The signing algorithm tags each message for the inner BFT instance
with the prefix “0”, and each message for the outer DailyBFT with the prefix “1” to avoid namespace collision.

% Figure 3
\subsubsection{The mempool subprotocol}

Initializes $\mathsf{TXs}$ with 0 and keeps a track of incoming transactions with a Union set. On receiving a $\mathsf{propose}$ call, it adds the
transaction to log and communciates with gossip protocol. It also supports $\mathsf{query}$ method to return confirmed transactions. By keeping
track of transactions in a set, it purges the ones already confirmed.

% Figure 4
\subsubsection{Main Hybrid Consensus protocol}

A newly spawned node with an implicit message routing that carries with it $\mathsf{history}$ of the transcripts sent and received.
This interacts with the following components - Mempools, Snailchain, Preprocess, Daily Offchain Consensus, and on chain validation.

\subsection{Variant Day Length and Hybrid Committee Election}
\label{sec:election}
In \cite{pass2017hybrid}, instances of $\mathsf{BFT}$ committees are switched every a fixed period of time (with the $\mathsf{snailchain}$ as a logical clock).
And new committee is formed simply by the miners of the latest $csize$ number of blocks inside $\mathsf{snailchain}$. In our consensus design, we want to exploit
the intuition that, if the committee behaves well, we don't have to force them to switch, and therefore the overhead of switching committee could be prevented in
some situtations. On the other hand, this will raise difficulty for new nodes to get elected as a committee member if the previous committee keeps good records.
Therefore we still keep the design of forcibly switching the committee every fixed amount of time, but with a much lower frequency, (for example, the committee
will be switched every $K$ days). On the other hand, we incoporate the idea of authenticated complaints from Thunderella~\cite{pass2017thunderella} where the
slowchain can be used as the evidence of mishavioring of $\mathsf{BFT}$ committee. That is, whenever a committee misbehavior is detected from the
$\mathsf{snailchain}$, the next day starting point (not necessarily the $K$-th day) will trigger a forced switch.

More over, we will replace the committee election criterion. In Hybrid Consensus, committee members are chosen from the miners of the most recent
$\mathsf{snailchain}$ blocks. We decide to select committee members based on a hybrid criterion by stakes and randomness. To be more specific, we allow
full nodes to propose special $\mathsf{stake\_in}$ and $\mathsf{stake\_out}$ transactions to temporarily freeze their token assets. And whenever a committee
switch happens, we will elect $\theta \cdot \mathsf{csize}$ accounts based on their frozen stakes where $\theta \in [0,1]$ is a manual parameter.
And following the design of~\cite{gilad2017algorand}, we choose the rest $(1-\theta)\cdot \mathsf{csize}$ nodes based on the result of
VRF~\cite{micali1999verifiable} where the seed is determined by the seed used in previous committee selection, as well as the proposed randomness
from recent $\mathsf{csize}$ blocks. Different from Algorand~\cite{gilad2017algorand}, here we don't count stake weights for this part of selection.

Notice that the nodes that is chosen by random functions will have a high probability of not being online. For this reason, given the estimated online
rate $r_{on}$, since we don't want the offline nodes to take the Byzantine quota, we need to make sure $r_{on}\cdot \theta < \frac{f}{\mathsf{csize}}$.
Usually we take $\theta = \frac{f}{2 r_{on} \mathsf{csize}}$. Another potential design alternative is that we can allow offline nodes by design~\cite{pass2017sleepy}.

Note that a party in charge of overwhelming computation resources will be likely to manipulate the input of VRF, but that already goes beyond our security assumption.

\subsection{Application Specific Design}

In our consensus design, we keep aware of application specific requirbements and tailor for them under the conditions that the consisttency, liveness
and security properties are not compromised.

\subsubsection{Physical Timing Restriction}

Conventional consensus design by default allows miners / committee members / leaders to re-order transactions within a small timing window.
This raises a problem for some decentralized applications such as commercial exchanges where the trading fairness requires the timing order between
transactions to be carefully preserved, or otherwise malicious (or, even normal rational) participants will have the incentive to re-order transactions,
or even insert its own transactions, to gain extra profits. And this incentive will be magnified under high throughputs.

And what's even worse, such malicious re-ordering is impossible to distinguish because naturally network latency will cause re-ordering and such latencies
can only be observed by the receiver itself and therefore it has the final evidence of numbers regarding network latency.

To support decentralized advertisement exchanges, we try to reduce such problems by incoporating one more restriction called sticky timestamp.
More specifically, with a heuristic parameter $T_\Delta$, when proposing transactions, we require the client to put a physical timestamp $T_p$ inside
the metadata of the transactio, and this physical timestamp is signed together with the other parts of the transaction. Later when validators inside
$\mathsf{BFT}$ verify the transaction, it will do the following extra checks as shown in Algorithm~\ref{alg:timestamp}.
% before evaluating and verify the transaction (a.k.a., semantic checks).

\begin{figure*}
\begin{algorithm}[H]
\KwData{Input Transaction $\mathsf{TX}$}
\KwResult{A Boolean value that indicates whether the verification is passed}
$\mathsf{current\_time} \gets \mathsf{Time.Now()}$\;
\If{$|current\_time-\mathsf{TX.T_p}|>T_\Delta$}{
	return $\mathsf{false}$\; \tcp{if the time skew is too large, reject $\mathsf TX$.}
}
var $\mathsf{txn\_history=}$ new static dictionary of lists\;
\eIf{$\mathsf{txn\_history[TX.from]} == NULL$}{
	$\mathsf{txn\_history[TX.from]} == [TX]$\;
}{
	\eIf{$\mathsf{txn\_history[TX.from][-1].T_p}-\mathsf{TX.T_p} > 0$}{
		return $\mathsf{false}$\; \tcp{To make sure the transactions from the same node preserve timing order.}
	}{
		$\mathsf{txn\_history[TX.from].append(TX)}$\;
		return $\mathsf{true}$\;
	}
}
\caption{Extra Verification Regarding Physcal Timestamp}
\label{alg:timestamp}
\end{algorithm}
\caption{Pseudo-Code for Extra Verification}
\end{figure*}

At the stage of materializing logs inside $\mathsf{BFT}$, the leader will sort the transaction batch according to its physical timestamps and break ties
(though very unlikely) with the sequence number. Actually this step is not necessary because we can enforce the order later in the evaluation and verification.
But for simplicity, we put it here.

This set of modications give us several extra properties:

\begin{enumerate}
\item The order of transactions from any node $N_i$ is internally preserved according to their physical timestamps. Thus the sequence order of these transactions
is strictly enforced. This will get rid of the possibility of some malicious re-ordering that involves two transcations from the same node.
\item The order within a batch of transactions output by the $\mathsf{BFT}$ committee is strictly ordered by timestamps.
\item Nodes cannot manipulate fake physical timestamps because of the timing window restriction.
\end{enumerate}

One obvious disadvantage of this modificaion will be the reduction in terms of throughput due to aborting transactions when the parameter $T_\Delta$ is inappropriate
for the varying network latency. Another disadvantage is that, the $\mathsf{BFT}$ committee members are still allowed to lie about their local time and reject certain
transactions. However, committee members can reject certain transactions anyway. But honest nodes could potentially reject ignorant transactions because of their
unsynchronzied clocks. This issue can be reduced by adding restrictions on the eligibilities of the $\mathsf{BFT}$ committee. Later we will see that to get into
the committee, the nodes should present evidence of synchronized clocks.

\subsection{Computation and Data Sharding, and Speculative Transaction Execution}
In this subsection we introduce our sharding scheme.

%\subsubsection{High Level Idea}
An important modification over the original Hybrid Consensus is that we add computation and data sharding support for it. And even more, first of its kind,
we design a speculative transaction processing system over shards. The idea is clear. In Hybrid Consensus, the $\mathsf{DailyBFT}$ instances are indexed into
a determininstic sequence $\mathsf{DailyBFT[1\dots R]}$. We allow multiple sequences of $\mathsf{DailyBFT}$ instances to exist at the same time. To be precise,
we denote the $t$-th $\mathsf{DailyBFT}$ sequence by shard $S_t$. For simplicity, we fix the number of shards as $C$. Each $\mathsf{DailyBFT}$ is a normal shard.
Besides $C$ normal shards, we have a primary shard $S_p$ composed of $\mathsf{csize}$ nodes. The job of the primary shard is to finalize the ordering of the
output of normal shards as well as implementing the coordinator in distributed transaction processing systems. And the normal shards instead of directly connects
with Hybrid Consensus component, it submit logs to the primary shards and then primary shards talks to Hybrid Consensus.

We don't allow any two shards (either normal or primary) to share common nodes, which can be enforced in the committee selection procedure. The election of
multiple shards is similar to the election procedure described in Section~\ref{sec:election}.

We partition the state data (in terms of account range) uniformly into $C$ shards. This will make sure that every query to the corresponding shard will return
a consistent state. Since we are going to include meta data for each data unit, we split data into units of data sectors and assign each data sector with an
address. We have a mapping from data position to data sector address. For simplicity, from now on, we only discuss at the level of data sectors. Each
data sector $\mathsf{DS[addr]}$ has metadata of $\mathsf{rts, wts, readers, writers}$.

We assume the partition principle is public and given the address $\mathsf{addr}$ we can get its host shard by calling the function $\mathsf{host(addr)}$.

Notice that if we treat every normal shard (when the number of adversaries is not large) as a distributed processing unit, we can incorporate the design of
logical timestamps \cite{yu2016tictoc} in distributed transaction processing systems \cite{mahmoud2014maat}, which will empower the processing of transactions.
Here we utilized a simplified version of MaaT where we don't do auto-adjustment of other transaction's timestamps.

For normal shards, it acts exactly as described in $\mathsf{DailyBFT}$ except the following changes to make it compatible for parallel speculative execution.

\begin{figure*}
\begin{algorithm}[H]
\textbf{On BecomeShard:}\\
\myin Initialize all the state data sectors: $\mathsf{lastReaderTS = -1, lastWriterTS = -1, readers=[], writers=[]}$

\textbf{With transaction $\mathsf{TX}$ on shard $S_i$:\\
On Initialization:}\\
\hspace{0.1in} $\mathsf{TX.lowerBound = 0}$\;
\hspace{0.1in} $\mathsf{TX.upperBound = +\infty}$\;
\hspace{0.1in} $\mathsf{TX.state = RUNNING}$\;
\hspace{0.1in} $\mathsf{TX.before = []}$\;
\hspace{0.1in} $\mathsf{TX.after = []}$\;
\hspace{0.1in} $\mathsf{TX.ID = rand}$\;
%\hspace{0.1in} $\mathsf{TX.id = a random}$\;

\textbf{On Read Address($\mathsf{addr}$):}\\
\eIf{$\mathsf{host(addr)}==S_i$}{
	Send $\mathsf{readRemote(addr)}$ to itself\;
}{
	Broadcast $\mathsf{readRemote(addr, TX.id)}$ to $\mathsf{host(addr)}$\;
	%Broadcast $\mathsf{readRemote(addr)}$ to $\mathsf{host(addr)}$\;
	Async wait for $2f+1$ valid signed replies within timeout $T_o$\;
	Abort $\mathsf{TX}$ when the timeout ticks\;
}
Let $\mathsf{val, wts, IDs}$ be the majority reply\;
$\mathsf{TX.before.append(IDs)}$\;
$\mathsf{TX.lowerBound= max(TX.lowerBound, wts)}$\;
return $\mathsf{val}$\;

\textbf{On Write Address($\mathsf{addr}$):}\\
\eIf{$\mathsf{host(addr)}==S_i$}{
	Send $\mathsf{writeRemote(addr)}$ to itself\;
}{
	Broadcast $\mathsf{writeRemote(addr, TX.id)}$ to $\mathsf{host(addr)}$\;
	%Broadcast $\mathsf{writeRemote(addr)}$ to $\mathsf{host(addr)}$\;
	Async wait for $2f+1$ valid signed replies within timeout $T_o$\;
	Abort $\mathsf{TX}$ when the timeout ticks.
}
Let $\mathsf{rts, IDs}$ be the majority reply\;
$\mathsf{TX.after.append(IDs)}$
$\mathsf{TX.lowerBound= max(TX.lowerBound, rts)}$\;
return\;

\textbf{On Finish Execution:}
\For{every $\mathsf{TX'} in TX.before$}{
	TX.lowerBound = max(TX.lowerBound, TX'.upperBound)\;
}
\For{every $\mathsf{TX'} in TX.after$}{
	TX.upperBound = min(TX.upperBound, TX'.lowerBound)\;
}
\If{TX.lowerBound > TX.upperBound}{
	Abort TX\;
}
Broadcast $Precommit(TX.ID, \lfloor \frac{TX.lowerBound + TX.upperBound}{2}\rfloor)$ to all the previous remote shards  which $TX$ has accessed\;
\tcp{If TX.upperBound = $\infty$, we can set an arbitrary number larger than $TX.lowerBound$.}

\textbf{On receive $\mathsf{readRemote(addr, ID)}$:}\\
\eIf{$\mathsf{host(addr)}==S_i$}{
	$\mathsf{DS[addr].readers.append(ID)}$\;
	return $\mathsf{DS[addr].value, DS[addr].wts, DS[addr].writers}$\;
}{
	Ignore
}

\textbf{On receive $\mathsf{writeRemote(addr, ID)}$:}\\
\eIf{$\mathsf{host(addr)}==S_i$}{
	$\mathsf{DS[addr].writers.append(ID)}$\;
	Write to a local copy\;
	return $\mathsf{DS[addr].rts, DS[addr].readers}$\;
}{
	Ignore
}

\caption{Sharding and Speculative Transaction Processing}
\label{alg:shard}
\end{algorithm}
\caption{Pseudo-Code for Sharding and Speculative Transaction Processing}
\end{figure*}

\begin{figure*}
\begin{algorithm}[H]
\textbf{On receive $\mathsf{Precommit(ID, cts)}$:}\\
Look up TX by ID\;
\If{Found and $\mathsf{cts}$ not in $\mathsf{[TX.lowerBound, TX.upperBound]}$}{
	Broadcast $Abort(ID)$ to the sender's shard.\;
}
TX.lowerBound = TX.upperBound = cts\;
For every data sector $DS[addr]$ $\mathsf{TX}$ reads, set $DS[addr].rts = max(DS[addr].rts, cts)$\;
For every data sector $DS[addr]$ $\mathsf{TX}$ writes, set $DS[addr].wts = max(DS[addr].wts, cts)$\;
Broadcast $Commit(ID, batchCounter)$ to the sender's shard.\;
\tcp{batchCounter is a number which increases by 1 whenever the shard submit a batch of log to the primary shard.}

\textbf{On receive $2f+1$ $\mathsf{Commit(ID, batchCounter)}$ from each remote shards which $TX$ has accessed:}\\
TX.lowerBound = TX.upperBound = cts\;
For every data sector $DS[addr]$ $\mathsf{TX}$ reads, set $DS[addr].rts = max(DS[addr].rts, cts)$\;
For every data sector $DS[addr]$ $\mathsf{TX}$ writes, set $DS[addr].wts = max(DS[addr].wts, cts)$\;
Mark $TX$ committed\;
Let $TX.metadata = [ShardID, batchCounter]$\;
%Let $TX.shards$ be the set of all the indices of remote shards it has accessed.

\textbf{On output log}\\
Sort $TX$'s based on their $cts$. Break ties by physical timestamp.

\caption{Sharding and Speculative Transaction Processing (cont.)}
\end{algorithm}
\end{figure*}

For the primary shard, it collects output from all the normal shards. Notice that, the data dependency of transactions can be easily inferred by their metadata.
And a fact is that, if a transaction visits multiple remote shards, it will leave traces in all the shards involved. When a normal shard submit logs to the primary
shard, it will also write to the snailchain.

When the primary shard receives (or fetch from the snailchain) a batch of txns from a shard, it will check if it has received from all the shards transactions
within this batch. If after certain timeout it has not received transactions from a particular batch, it means that batch has failed. In this case, a whole committee
switch will be triggered at the next day starting point. After receiving all the shards' logs, the primary shard sort the transactions based on their commit
timestamps (if some transaction has earlier batch number, it will be considered as the first key in the sorting, however, if its physical timestamp violates
the timestamps from many shards, we decide that batch as invalid and all the transactions inside that batch are aborted). After sorting the primary shards
filters all the transactions and keep a longest non-decreasing sequence in terms of physical timestamps. Out the log to the Hybrid Consensus component as that
day's log.

There are still many optimisation spaces. One certain con is that the confirmation time in this design is not instant.

\section{Smart Contracts in Virtual Machines}

\subsection{Design Rationale}

Of all the reasons to have an Ethereum Virtual Machine (EVM)~\cite{gavinethereum}, one of aims is to meter the usage with a transaction fee
in a Proof of Work model. Since ours is a hybrid model, we'll take the liberty of exploring this design space a little bit further. Let us
consider the possibility of a hybrid cloud ecosystem.

A basic problem people have faced is the kind of crude mathematical notations followed in the Ethereum yellowpaper~\cite{gavinethereum}.
We therefore hope to follow something like KEVM jellopaper~\cite{kevmjello} to list out the EVM and TVM (described in~\ref{sec:tvm}) specifications.
And in future, we hope to maintain our own specifications Truechain's github account (https://github.com/truechain). 

\subsubsection{What about containers instead of VMs?}

One of the blockchain frameworks out there that come as close to this idea as possible, is Hyperledger's Fabric framework~\cite{fabricpaper}.
If one sets out to convert Fabric's permissioned nature into permissionless, one of the foremost challenges
would be to solve the chaincode issue. What this means is while it's possible to keep a chaincode/smart contract in a single container,
that is not a scalable model for a public chain. Having such a model for public chain means having to run several thousand containers,
per se, several thousand smart contracts on a single node (because each node maintains a copy).
% \linebreak

\smallskip
There have been attempts from the community on being able to run a certain mainimum containers per node. The limit currently is 100 pods
per node, per se, approxumately 250 containers per node, as illustrated in Kubernetes container orchestration platform~\cite{kubelimit}
and Red Hat's Openshift Container Platform 3.9's Cluster Limits~\cite{ocpclusterlimit}. Even with latest storage techniques like
brick multiplexing~\cite{brickmul}, the max possible value (say $\mathsf{MAX\_CONTR}$) of containers could not possibly reach
(at least right now) 1000. This issue could further be looked up in the discussions on kubernetes issues
github page~\cite{k8scale} around workload-specific limits that usually determine the maximum pods per node.
People who wish to scale containers usually prefer horizontal scaling rather than a vertical scaleup~\cite{cncfscaleout, kubecommgoals},
as the latter significantly increases complexity of design decisions. And there's no one-size-fits-them-all rule for a cluster scale
configuration as that entirely depends on the workload, which being more in our case due to its decentralized nature, isn't very
convincing for taking a step towards scaling this. At this point, it becomes more of an innovation problem than a simple technical
specification search. Ethereum currently has $\mathsf{ > 1000 }$ smart contracts deployed. Therefore this would be nothing but a crude
attempt at optimizing the container ecosystem's design space.

Now let's expand a bit on the container scenario. Given the above crisis, a possible solution is to use container in a
serverless architecture. But consider a scenario where $\mathsf{> 2000}$ contracts are online and the concurrent requests,
i.e., invocation calls to chaincode (a moving window) at a time exceed $\mathsf{MAX\_CONTR}$ value, we then face the same
problem all over again. Therefore, it's only advisable to add a throttling rate limit on the max concurrent requests.
This severly limits the Transactions Per Second from the consensus, by design. Engineering should not be a bottleneck to what could
be achievable alternatively. Therefore, we choose to stick to EVM design, although a bit modified for our purpose.

\subsection{Truechain Virtual Machine (TVM)}
\label{sec:tvm}

A typical example in this space would be that of the Ethereum Virtual Machine (EVM) ~\cite{gavinethereum}, which tries to follow
total determinism, is completely optimized and is as simple as it gets, to make incentivization simple step to calculate.
It also supports various features like off-stack storage of memory, contract delegation and inter-invocation value storage.

We would reuse the EVM specifications for the snailchain, but add a new specification for TVM in the next version of this yellowpaper,
after careful consideration of the design rationale similar to EVM, deriving the stack based architecture utilizing the Keccak-256
hashing technique and the Elliptic-curve cryptography (ECC) approach.

The Truechain infrastructure will utilize a combination of EVM and another EVM-like bytecode execution platform for launching smart contracts.
We choose to use one VM for POW and another for PBFT, embedded within each full node, so they could manage invocation calls on per-need basis.

The TVM backs the DailyBFT powered chains, which interact with the following components:
\begin{itemize}
  \item re-using some of the concepts from tendermint, like the ABCI (Application BlockChain Interface) which offers
  an abstraction level as means to enable a consensus engine running in one process to manage an application state running in another.
  \item A different consensus engine pertaining to dailyBFT chians,
  \item A permissioned Ethereum virtual machine
  \item An RPC gateway, which guarantees (in a partially asynchronous network) transaction finality
\end{itemize}

% The TVM deals with the following state sets:
\#TODO - formally define transition states of TVM, smart contracts deployment strategy
and a way to deploy permissioned VM onto a permissionless chain.

\#TODO - define parameter to switch between the POW and the full node (POW and PBFT).

\section{Blocks, State and Transactions}

\# TODO - Talk about changes to blocks, world state flow, transactions and execution model
  
% \subsection{The world state and snapshotting}
% \subsection{Homestead}
% \subsection{Transactions and execution}
% \subsection{The Block}

\section{Changes to the Ethereum Blockchain Paradigm}

\subsection{Fruitchain as snailchain}

The greatest challenge that POW based consensus faces today are efficiency and scalability. Slow execution and confirmation time make it unfit to develop complex applications, and excessive energy consumption make it environmentally unfriendly. The protocol we propose use a hybrid of PBFT and PoW as core consensus. Unlike Ethereum, where transactions and smart contracts are executed by every node on the network, a rotating BFT committee will handle the bulk of heavy lifting while PoW (the snailchain) will only be used to select the committee members. Observe that in the limiting case, where we take committee rotating frequency to one block and $\mathrm{csize} = 1$, we recover the traditional PoW consensus. 

A functioning BFT committee require 2/3 of its members to be honest\cite{castro1999practical}. Hence, for some $\epsilon > 0$, we require the chain quality $Q > 2/3 + \epsilon$. A naive implementation using a Nakamoto chain as the snailchain will be vulnerable to obvious selfish mining attack strategies. If a selfish miner controlled more than 25\% of the blockchain's hash power, she could control more than 33\% of block production\cite{naya2015stub}\cite{eyal2013self}. The probability of being elected to the BFT committee, according to the procedure described in \cite{pass2017hybrid}, is equal to one's block production fraction. Hence, the selfish miner is likely to control over 1/3 of the BFT committee, and thus the BFT protocol is compromised. 

The worst case scenario is possible through a strategy illustrated in \cite{ritz2018uncle}. If a selfish miner controls 40\% of the hash power in a Ethereum-type blockchain, she can control 70\% of block production through optimized selfish mining. According to the committee election procedure in \cite{pass2017hybrid}, she will have control over 70\% of the BFT committee. The BFT committee is not only compromised, the selfish miner will have the dictatorial dominance to declare any honest committee member as 'dishonest' at her will. 

We choose the fruitchain\cite{pass2017fruit} as our underlying snailchain for hybrid consensus. The fruitchain is more resistant to selfish mining, as stated by the fairness theorem in \cite{pass2017fruit}. However, the BFT committee is still vulnerable should an attacker directly control over 33\% of the blockchain's hash power. Hence, we will take further deviations from \cite{pass2017hybrid} and \cite{pass2017fruit} to alleviate the issues. We also remark that given the current market share of mining pools, it's not at all inconceivable for a party to command significant proportion of hash power in a blockchain. 

There are two undesirable extremes that we need to find a balance in, 
\begin{itemize}
	\item Randomly select BFT members via VRF\cite{micali1999verifiable}. This is vulnerable against a Sybil attack.
	\newline
	\item Any selection procedure where probability of being selected is proportional to hash power. The BFT committee is vulnerable against mining pools who are rich in hash power. 
\end{itemize} 


Our proposed solution are as follows. When an honest BFT node's chain reaches $\lambda$ in length, it will publish the unique miner IDs of every fruit in the chain as candidates (or, every miner ID who mined more than $\nu$ fruits). The new BFT committee is randomly selected from the candidates though application of VRF.  

The obvious Sybil attack is not possible under this scheme, as you require a minimal level of PoW to become a candidate. Neither it will be easy for a large mining pool to achieve compromise the BFT committee. A fruit will be orders of magnitude easier to mine than a block. 


\subsection{Incentive Design}

Inheriting the variable definitions from \cite{pass2017fruit} (p.14 - 16), the fruitchain consist of a blockchain with each block containing its own set of fruits. Transactions executed by the BFT will be initially packaged as a record $\mathrm{m}$ to be mined as a fruit. Fruits more recent than a recency parameter $R$ will be packaged into a block when the next block is mined. 

The miner will run only one mining algorithm that produces hash values $h$ from a random oracle. A fruit is mined when $[h]_{-\kappa} < D_{p_f}$, and a block is mined when $[h]_{\kappa} < D_{p}$, where $D_{p_f}$ and $D_p$ are the mining difficulty parameter of the fruit and block respectively. The tuple $(R, D_p, D_{p_f})$ determines mining process. 

In order to discourage the deployment of ASICs, we will make the recency parameter $\kappa = \kappa (t)$ time-dependent. VRF will generate and broadcast a new $\kappa (t)$ (to fall within the valid range) using VRF once every 3 months. Details of this process will be included in a future version of the yellow paper. 


More specifically, the mining algorithm goes as follows. A fruit is a tuple $f = (h_{-1}; h'; \eta ; \mathrm{digest}; m; h)$, where each entry means

\begin{itemize}
	\item $h_{-1}$ is only useful for fruit verification.
	
	\item $h'$ points to a block that contains the fruit.
	
	\item $\eta$ is the random nonce. 
	
	\item $\mathrm{digest}$ is a value used to check the fruit's validity. 
	
	\item $\mathrm{m}$ is the record contained in the fruit.
	
	\item $h$ is the hash value of the fruit.	
\end{itemize}

A block is a tuple $\mathrm{b} = ((h_{-1}; h'; \eta; \mathrm{digest}; \mathrm{m}; h), F)$, where 
\begin{itemize}
	\item $h_{-1}$ points to the previous block's reference.
	
	\item $h'$ is only useful for block verification. 
	
	\item $\eta$ is the random nonce. 
	
	\item $\mathrm{digest} = \mathrm{d}(F)$, where $\mathrm{d}$ is a collision resistant hash function, and $F$ is the fruit-set to be included in the block. 
	
	\item $\mathrm{m}$ is the record contained in the fruit.
	
	\item $h$ is the hash value of the fruit.	
\end{itemize}

The blockchain $chain$ will grow as follows. We initialize by setting the genesis block as,
$$
chain[0] = (0;0;0;0;\perp ;H(0;0;0;0;\perp ),\emptyset )
$$
and $F = \emptyset$. Every time step, upon receiving input $\mathrm{m}$ from the environment, 
\begin{itemize}
	\item Let $F'$ be all fruits $f \in F$ that are recent w.r.t. ¥$chain$ and not already in $chain$.
	
	\item Let $h'$ be the reference of $chain[pos]$ where $pos = \max ( 1, |chain| - \kappa ) $. 
	
	\item Let $h_{-1}$ be the reference of $chain[|chain| - 1]$. 
	
	\item Pick random nonce $\eta \in \{ 0,1 \}^\kappa$ and let $h:=H(h_{-1}; h';\eta ; d(F'); \mathrm{m})$. 
	
	\item If $[h]_{-\kappa} < D_{p_f}$ (we mined a fruit), then 
	\begin{itemize}
	\item $fruit:=(h_{-1}; h';\eta ; d(F'); \mathrm{m};h)$
	
	\item $F := F \cup \{ fruit \}$
	
	\item broadcast
	\end{itemize}

	\item If $[h]_{\kappa} < D_{p}$ (we mined a block), then 
\begin{itemize}	
	\item $chain := chain || ((h_{-1}; h';\eta ; d(F'); \mathrm{m}; h), F')$
	
	\item broadcast
\end{itemize}

	\item $\mathrm{digest} = \mathrm{d}(F)$, where $\mathrm{d}$ is a collision resistant hash function, and $F$ is the fruit-set to be included in the block. 
	
	\item $\mathrm{m}$ is the record contained in the fruit.
	
	\item $h$ is the hash value of the fruit.	
\end{itemize}

\subsection{Incentive Design}

There are in total 30 million $\mathrm{TRUE}$ coins, of which 24 million are minable.

Typically, a fruit will be released once every second, and a block once every 10 minutes. 


\subsection{Compensation Infrastructure}

In this section we will present a concept of composation infrastructure in order to balance the workload of $\mathsf{BFT}$
committee members and non-member full nodes.

Treating all shards as equivalent of each other in terms of network and CPU bandwidth could produce skewed results,
with inconsistent TPS, or worse, sometimes cross timeout limits, while ordering of transaction takes place from the
Primary shard. To tackle this, we propose a compensation infrastructure, that works along the lines of Berkeley Open
Infrastructure for Network Computing. There has been a previous attempt in this area from Gridcoin~\cite{gridwhitepaper}
and Golem network~\cite{golemwhitepaper}.

Gridcoin's distributed processing model relies pre-approved frameworks to the like of Berkeley Open Infrastructure
for Network Computing (BOINC)~\cite{boincproject}, an opensource distributed volunteer computing infrastructure,
heavily utilized within cernVM\cite{cernvm} in turn, harnessed by the LHC@Home project~\cite{lhcathome}
A framework like this has to tackle non-uniform wealth distribution over time. On the other hand, Golem is another great
ongoing project with concrete incentivization scheme, which would be used as an inspiration for compensation infrastructure's
incentivization methodology. However a keeping in mind, a widely known problem is that a blockchain powered volunteer computing
based rewarding model could easily fall prey to interest inflation if the design lacks a decent incentive distribution scheme over time.
So to speak, an increasing gap between initial stake holders minting interest due to beginner's luck (algorithmic luck) and the contributors
joining late, could thence be found fighting for rewards from smaller compensation pools that further condense.

Depending on the kinds of transactions and whether we'd need decentralized storage for some of the smart contracts, we propose the use of
a hybrid infrastructure that utilizes BOINC and IPFS/Swarm, along side of EVM and TVM. This would make use of Linux Containers to deal with
isolation of resources and we hope to expand on this section in the next version of this yellowpaper.

% \input{sections/interchain_communications}
\section{Future Direction}

Even after optimizations to the original Hybrid Consensus, we acknowledge various optimizations possible on top of what was
proposed in this paper. There's possibility of:

\begin{itemize}
  \item Improving timestamp synchronization for all nodes, with no dependency on centralized NTP servers.
  \item Detailed incentivization techniques for compensation infrastructure, so heavy infrastructure investors don't suffer from 'left-out', 'at a loss' problem
  \item Sharding techniques with replica creation to minimize the transaction set rejection from the BFT committee.
  \item Addition of zero knowledge proof techniques for privacy.
  \item Hybrid infrastructure of EVM, TVM and Linux container ecosystem.
  \item Sections for Virtual Machine Specification, Binary Data Encoding Method, Signing Transactions, Fee Schedule and Ethash Alternative.
\end{itemize}

\section{Conclusions}

We have formally defined Hybrid Consensus protocol and its implementation along with plausible speculations in the original
proposal as well as implementation specifications. In this draft of paper, we have introduced various new concepts
some of which we will detail in the next version very soon. We recommend people to choose ASIC resistant hardware for
deployment of the POW only versus full nodes, although more details on hardware shall follow soon. 

\section{Acknowledgements}

We owe a great deal of appreciation and are thankful, to the following folks for their untiring work towards pushing the protocols
for a decentralized sovereignty, for their design rationale and implementations that serverd as a solid reference architecture
in our proposal above. These folks and their legacies are as mentioned below:

\begin{itemize}
  \item Rafael Pass, Miguel Castro, Satoshi Nakamoto, Vitalik Buterin, Gavin Wood, Ethan Buchman, Andrew Miller et al
  for their untiring work, contributions and continuous improvisations while spearheading the glamorous Improvement Proposals forums in
  addition to the active participation through Reddit, Mailing lists, chat forums, white and Yellow Papers, and rest of the mediums alike.
  \item CNCF and Kubernetes communities for their inspiring ventures into hybrid cloud computing.
  % \item For their timely support from an engineering perspective:- Vipin Rathi, Akshay Arora, and a few Red Hat DNA mutants
  % [Sathish VJ, Raghavendra Talur, Shekhar Berry, Sathish Mohan]
  % \item All those involved with Ethereum, Tendermint, Quorum and Hyperledger projects.
\end{itemize}


% random number
% data feeds / oracles

\bibliography{ref}
\bibliographystyle{abbrv}

\end{multicols}

\appendix

\section{Terminology} \label{ch:Terminology}

\begin{description}
\item[TrueChain Virtual Machine (TVM)] In contrast to EVM which does handles incentivization and rotating committee selection,
  a TVM is based on similar design principles but carries out actual consensus and voting based off of PBFT based Hybrid Consensus.
\end{description}

% \section{Virtual Machine Specification} \label{ch:Virtual Machine Specification}
% \section{Binary Data Encoding Method} \label{ch:Binary Data Encoding Method}
% \section{Signing Transactions} \label{ch:Signing Transactions}
% \section{Fee Schedule} \label{ch:Fee Schedule}
% \section{Ethash Alternative} \label{ch:Ethash Alternative}

\end{document}
